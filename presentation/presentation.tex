\documentclass{beamer}

\usepackage{pgfpages}
\setbeameroption{hide notes} % Only slides
% \setbeameroption{show only notes} % Only notes
%\setbeameroption{show notes on second screen=right}% Both

\usepackage{hyperref}
\usepackage{amsfonts}
\usepackage{amssymb}
\usepackage{amsmath}
\usepackage{stmaryrd}
\usepackage{color}

\usepackage{bussproofs}

\usepackage{colonequals}

\usepackage{listings}

\usepackage[all,cmtip]{xy}

\usetheme{default}
\usecolortheme{default}

\lstset{escapeinside={<@}{@>}}
\setbeamertemplate{sidebar right}{}
\setbeamertemplate{footline}{%
	\hfill\usebeamertemplate***{navigation symbols}
	\hspace{1cm}\insertframenumber{}/\inserttotalframenumber}

\title{The Construction of Set-Truncated Higher Inductive Types}
\author{\textbf{Niels van der Weide} \and Herman Geuvers}
\institute{Radboud University}
\date[MFPS2019]{June 7, 2019}

\newcommand{\remove}[1]{}

\newcommand{\sets}{\mathbf{Set}}
\newcommand{\setoids}{\mathbf{Setoid}}
\newcommand{\quotient}{\mathbf{Quot}}
\newcommand{\paths}{\mathbf{Path}}

\newcommand{\prealgsets}{\mathbf{PreAlgSet}}
\newcommand{\prealgsetoids}{\mathbf{PreAlgSetoid}}
\newcommand{\prealgquotient}{\mathbf{PreAlgQuot}}
\newcommand{\prealgpaths}{\mathbf{PreAlgPath}}

\newcommand{\algsets}{\mathbf{AlgSet}}
\newcommand{\algsetoids}{\mathbf{AlgSetoid}}
\newcommand{\algquotient}{\mathbf{AlgQuot}}
\newcommand{\algpaths}{\mathbf{AlgPath}}

% Endpoints
\newcommand{\ep}[3]{\mathcal{E}_{#1}(#2,#3)} % type of endpoinst
\newcommand{\id}[1]{\constructor{id}_{#1}} % identity endpoints
\newcommand{\comp}[2]{#1 \cdot #2} % composition of endpoints
\newcommand{\inle}{\constructor{inl}} %left inclusion of endpoints
\newcommand{\inre}{\constructor{inr}} % right inclusion of endpoints
\newcommand{\prle}{\constructor{pr}_1} % first projection of endpoints
\newcommand{\prre}{\constructor{pr}_2} % second projection of endpoints
\newcommand{\pair}[2]{(#1 , #2)} % pairing of endpoints
\newcommand{\Ce}{\constructor{c}} % constant endpoint
\newcommand{\constr}{\constructor{constr}} % constructor endpoint
\newcommand{\fmap}{\constructor{fmap}} % function endpoint

\newcommand{\poly}{\mathcal{P}} % type of polynomials
\newcommand{\C}{\constructor{C}} % constant polynomial
\newcommand{\I}{\constructor{I}} % identity polynomial
\newcommand{\sumP}[2]{#1 + #2} % sum of polynomials
\newcommand{\prodP}[2]{#1 \times #2} % product of polynomials

\newcommand{\hset}{\type{hSet}} % sets

\begin{document}
\beamertemplatenavigationsymbolsempty

\frame
{
	\maketitle
}

\begin{frame}
\frametitle{Homotopy Type Theory: Another Perspective on Types}
Its features are:
\begin{itemize}
	\item Equality is proof relevant (\emph{goodbye UIP})
	\item<2,3,4> Interpretation in simplicial sets (\emph{types are spaces, terms are points, equalities are paths})
	\item<3,4> Univalence Axiom (\emph{equality of types is isomorphism})
	\item<4> \textbf{Higher Inductive Types} (\emph{building spaces})
\end{itemize}
\end{frame}

\begin{frame}
\frametitle{Higher Inductive Types? What are they?}
Inductive Types: a type generated by constructors for its points

For example, $\mathbb{N}$ is generated by $Z : \mathbb{N}$ and $S : \mathbb{N} \rightarrow \mathbb{N}$.
\end{frame}

\begin{frame}
\frametitle{Higher Inductive Types? What are they?}
\textbf{Higher} Inductive Types (HIT): a type generated by constructors for its points \textbf{and paths} (equalities)

For example, $X$ is generated by $Z : X$ and $S : X \rightarrow X$ \textbf{and}
\[
m : \prod_{n : X} n = S(S \> n).
\]

\pause

Now we have two ways to prove $S \> Z = S^3 \> Z$:
\begin{itemize}
	\item use $m(S \> Z)$
	\item use $m(Z)$ and $S$ preserves equality
\end{itemize}
\end{frame}

\begin{frame}
\frametitle{Higher Inductive Types? What are they?}
\textbf{Higher} Inductive Types (HIT): a type generated by constructors for its points, paths (equalities), \textbf{and homotopies (equalities of equalities)}

For example, $\mathbb{N}_2$ is generated by $Z : \mathbb{N}_2$ and $S : \mathbb{N}_2 \rightarrow \mathbb{N}_2$ \textbf{and two equalities:}
\[
m : \prod_{n : \mathbb{N}_2} n = S(S \> n),
\]
\[
t : \prod_{n, m : \mathbb{N}_2} \prod_{p, q : n = m} p = q.
\]
\end{frame}

\begin{frame}
\frametitle{But let's not get too high}
In this talk, we only consider HITs
\begin{itemize}
	\item constructed by giving points and equalities
	\item and with the constructor
	\[
	t : \prod_{x, y : X} \prod_{p, q : x = y} p = q.
	\]
\end{itemize}
We call these \textbf{set-truncated HITs}.

Recall: a set is a type for which all $p, q : x = y$ are the same.
\end{frame}

\begin{frame}
\frametitle{Now, what did we do?}
Our result: all set-truncated HITs exists if two simple ones exist:
\begin{itemize}
	\item The \textbf{quotient}. Given $A$ and an equivalence relation $R$ on $A$, identify the points in $A$ via $R$.
	\item Given a type $A$, then the \textbf{propositional truncation} $||A||$ is $A$ with all its points identified.
\end{itemize}
Note: we construct them in HoTT. Formalization in UniMath.
\end{frame}

\begin{frame}
\frametitle{A Bird's-Eye View of our Construction}
The steps:
\begin{itemize}
	\item Define signatures of HITs (\emph{how to describe them})
	\item Give categories of algebras in sets \textbf{and setoids} (\emph{the introduction rules})
	\item Use initial algebra semantics (\emph{showing induction by initiality})
	\item Relate those categories by an adjunction (\emph{the main idea})
	\item Construct the initial algebra in setoids (\emph{where we get to work})
\end{itemize}
Recall:
\begin{itemize}
	\item a set is a type for which all proofs of equality are equal
	\item a setoid is a set with an equivalence relation
\end{itemize}
\end{frame}

\begin{frame}
\frametitle{Wells of Inspiration}
Note
\begin{itemize}
	\item Initial algebra semantics is used: QIITs (by Altenkirch, Capriotti, Dijkstra, Kraus, Forsberg) and W-suspensions (Sojakova).
	\item To obtain the adjunction, we use a result by Hermida and Jacobs.
	\item The construction of the initial algebra in setoids is an adaption of work by Dybjer and Moenclaey.
\end{itemize}
\end{frame}

\begin{frame}
\frametitle{Let's Get Started: Signatures for HITs}
Recall: $\mathbb{N}_2$ is generated by $Z : \mathbb{N}_2$ and $S : \mathbb{N}_2 \rightarrow \mathbb{N}_2$ and two equalities:
\[
m : \prod_{n : \mathbb{N}_2} n = S(S \> n),
\]
\[
t : \prod_{n, m : \mathbb{N}_2} \prod_{p, q : n = m} p = q.
\]

To define signatures of HITs, note the following:
\begin{itemize}
	\item We must describe arguments of the point constructors
	\item We must describe the possible endpoints for the equalities, for which we can refer to the point constructor
\end{itemize}
Then the introduction/elimination rules are derived.
\end{frame}

\begin{frame}
\frametitle{First Ingredient: Point Constructors}
Same idea as for inductive types.

We use \textbf{finitary polynomials}.
These are generated by
\begin{itemize}
	\item The \emph{identity}
	\item Given a set $X$, we have a \emph{constant} polynomial on $X$
	\item Given two polynomials, we have their \emph{product} and \emph{sum}
\end{itemize}
Recall $\mathbb{N}$ generated by $Z : \mathbb{N}$ and $S : \mathbb{N} \rightarrow \mathbb{N}$.

This is represented by $F(X) = 1 + X$
\end{frame}

\begin{frame}
\frametitle{Second Ingredient: Path Constructors}
Recall path of $\mathbb{N}_2$:
\[
m : \prod_{n : \mathbb{N}_2} n = S(S \> n),
\]
\pause
There are two ingredients for path constructors:
\begin{itemize}
	\item A polynomial representing the \emph{source} of the equation
	\item Two endpoints representing the \emph{sides} of the equation 
\end{itemize}
The endpoints represent all possible left- and right hand sides of equations.
Note: they \emph{depend on the point constructor and source}.
\pause
Main challenge: define tthe type of endpoints
\begin{itemize}
	\item it's a type $\mathcal{E}_P(S)$ depending polynomials $P, S$
	\item inductively generated by 10 constructors
\end{itemize}
\end{frame}

\begin{frame}
\frametitle{Putting it together: Signatures}
\begin{definition}
A \textbf{HIT signature} consists of
\begin{itemize}
	\item A polynomial $P$ representing its \emph{point constructor}
	\item A type $J$ representing \emph{labels for its path constructors}
	\item For each $j : J$, a polynomial $S_j$ representing the \emph{arguments of the path constructors}
	\item For each $j : J$, two endpoint with source $S_j$ using constructors from $P$ representing the \emph{path}
\end{itemize}
\end{definition}
Note: we will interpret signatures in sets, so we always have a constructor
\[
t : \prod_{n, m : X} \prod_{p, q : n = m} p = q.
\]
\end{frame}

\begin{frame}
\frametitle{How does this give rise to a notion of HITs?}
\begin{itemize}
	\item All we did so far, was say how to describe the constructors.
	\item Soon we discuss algebras, which describe the introduction rules.
	\item You also need to formulate an induction principle.
	\item Details on that are in the paper/formalization.
\end{itemize}
\end{frame}

\begin{frame}
\frametitle{Intermezzo: Some Examples}
\begin{itemize}
	\item Signature $\mathbb{N}_2$ (discussed before)
	\item Similarly, we can define a signature for the integers.
	\item We can define signatures of groups, rings, and so on
	\item Also: polynomials, free algebras
\end{itemize}
\end{frame}

\begin{frame}
\frametitle{A Bird's-Eye View of our Construction}
The steps:
\begin{itemize}
	\item Define signatures of HITs (how to describe them)
	\item \textbf{Give categories of algebras in sets and setoids (the introduction rules)}
	\item Use initial algebra semantics (showing induction by initiality)
	\item Relate those categories by an adjunction (the main idea)
	\item Construct the initial algebra in setoids (where we get to work)
\end{itemize}
\end{frame}

\begin{frame}
\frametitle{Algebra in Two Steps: the \textbf{pre}algebras}
Note: each polynomial $P$ gives rise to a functor $\overline{P}$ on $\sets$.

Now let $S$ be a signature and let $P$ be the point constructor.
\begin{itemize}
	\item A \textbf{prealgebra} on a signature consists of a set $X$ with a map $\overline{P} \> X \rightarrow X$.
	\item This forms a category $\prealgsets(P)$ whose morphisms are
	\[
	\xymatrix
	{
		\overline{P} \> X \ar[d] \ar[r]^{P \> f} & \overline{P} \> Y \ar[d]\\
		X \ar[r]_f & Y 
	}
	\]
	\item Note: we have a functor $U : \prealgsets(P) \rightarrow \sets$
\end{itemize}
\end{frame}

\begin{frame}
\frametitle{Algebra in Two Steps: the \textbf{actual} algebras}
Now come the equations.

Let $e$ be an endpoint with source $S$, and constructor $P$.
Note we have functors:
\[
\xymatrix
{
	\prealgsets(P) \ar[r]^-{U} & \sets \ar[r]^{S} & \sets 
}
\]
Then $e$ gives a natural transformation from $S \circ U$ to $U$.

Algebras are defined as a full subcategory of prealgebras.
\end{frame}

\begin{frame}
\frametitle{We can repeat this story for setoids}
It will be much the same but with setoids instead.
\end{frame}

\begin{frame}
\frametitle{A Bird's-Eye View of our Construction}
The steps:
\begin{itemize}
	\item Define signatures of HITs (how to describe them)
	\item Give categories of algebras in sets and setoids (the introduction rules)
	\item \textbf{Use initial algebra semantics (showing induction by initiality)}
	\item Relate those categories by an adjunction (the main idea)
	\item Construct the initial algebra in setoids (where we get to work)
\end{itemize}
\end{frame}

\begin{frame}
\frametitle{The Induction Principle and Initial Algebras}
\begin{itemize}
	\item HITs satisfy an \emph{induction principle} (IP)
	\item The IP is formulated with \emph{displayed algebras} (aka fibred algebras)
	\item To verify IP, we use \emph{initial algebra semantics}
	\item It says: \textbf{induction follows from initiality}
\end{itemize}
\end{frame}

\begin{frame}
\frametitle{A Bird's-Eye View of our Construction}
The steps:
\begin{itemize}
	\item Define signatures of HITs (how to describe them)
	\item Give categories of algebras in sets and setoids (the introduction rules)
	\item Use initial algebra semantics (showing induction by initiality)
	\item \textbf{Relate those categories by an adjunction (the main idea)}
	\item Construct the initial algebra in setoids (where we get to work)
\end{itemize}
\end{frame}

\begin{frame}
\frametitle{The moment we waited for: Adjunctions}
Let's start with the ``base''.
Write
\begin{itemize}
	\item $\sets$ and $\setoids$ for the categories of sets and setoids respectively
	\item $\quotient$ sends a setoid $(X,R)$ to the quotient of $X$ by $R$
	\item $\paths$ sends a set $X$ to the setoid $(X, =)$
\end{itemize}
Then we have an adjunction
\[
\xymatrix
{
	\sets \ar@/^/[rrrrr]_-{\top}^-{\paths} & & & & & \ar@/^/[lllll]^-{\quotient} \setoids
}
\]
\end{frame}

\begin{frame}
\frametitle{The moment we waited for: Adjunctions}
Lift it to an adjunction on prealgebras (Hermida and Jacobs)
\[
\xymatrix
{
	\prealgsets \ar@/^/[rrrrr]_-{\top}^-{\prealgpaths} \ar[d] & & & & & \ar@/^/[lllll]^-{\prealgquotient} \prealgsetoids \ar[d] \\
	\sets \ar@/^/[rrrrr]_-{\top}^-{\paths} & & & & & \ar@/^/[lllll]^-{\quotient} \setoids
}
\]
Here we need the polynomials are finitary.

For infinitary case, we need AC (Chapman, Uustalu, Veltri)
\end{frame}

\begin{frame}
\frametitle{The moment we waited for: Adjunctions}
Next we lift it to the level of algebras
\[
\xymatrix
{
	\algsets \ar@/^/[rrrrr]_-{\top}^-{\algpaths} \ar[d] & & & & & \ar@/^/[lllll]^-{\algquotient} \algsetoids \ar[d] \\
	\prealgsets \ar@/^/[rrrrr]_-{\top}^-{\prealgpaths} \ar[d] & & & & & \ar@/^/[lllll]^-{\prealgquotient} \prealgsetoids \ar[d] \\
	\sets \ar@/^/[rrrrr]_-{\top}^-{\paths} & & & & & \ar@/^/[lllll]^-{\quotient} \setoids
}
\]
\end{frame}

\begin{frame}
\frametitle{Before we continue, where are we?}
\begin{itemize}
	\item Define signatures of HITs (how to describe them)
	\item Give categories of algebras in sets and setoids (the introduction rules)
	\item Use initial algebra semantics (showing induction by initiality)
	\item Relate those categories by an adjunction (the main idea)
	\item \textbf{Construct the initial algebra in setoids (where we get to work)}
\end{itemize}

Recall that our goal was to construct HITs.
However, so far, no constructing actually happened.

We only did preparatory work, which shows that

\center{\textbf{We obtain HITs by constructing initial algebras in setoids!}}
\end{frame}

\begin{frame}
\frametitle{Finishing the proof: The Initial Algebra in Setoids}
\begin{itemize}
	\item Dybjer and Moenclaey give an interpretation of HITs in the setoid and groupoid model
	\item The set is inductively generated by the point constructor
	\item The equivalence relation is the inductive family generated by the path constructors
\end{itemize}
Here we need the propositional truncation.
\begin{itemize}
	\item In HoTT, equivalence relations take values in propositions
	\item Recall: a proposition is a type for which all inhabitants are equal
	\item To force the generated relation to be a proposition, we truncate it
\end{itemize}
\end{frame}

\begin{frame}
\frametitle{Time to Wrap Up}
\begin{theorem}
In UF with quotients and the propositional truncation, we can construct all finitary set-truncated HITs.
\end{theorem}
We obtain two consequences of our construction:
\begin{itemize}
	\item A uniqueness principle for HITs (\emph{go univalence})
	\item A characterization of the path space of HITs
\end{itemize}
\end{frame}

\begin{frame}
\frametitle{Where to go from here?}
Remove the truncatedness restriction.
\begin{itemize}
	\item First step: go to 1-types/groupoids
	\item Their structure is encapsulated by \textbf{bicategories}
	\item Imitate this construction bicategorically
\end{itemize}
Formalization: \url{https://github.com/nmvdw/SetHITs}.
\end{frame}

\begin{frame}
\frametitle{References}
\begin{enumerate}
	\item T. Altenkirch, P. Capriotti, G. Dijkstra, N. Kraus, and
	F.N. Forsberg.
	Quotient Inductive-Inductive Types.
	In FOSSACS 2018, pages 293--310.
	\item J. Chapman, T. Uustalu, and N. Veltri.
	Quotienting the Delay Monad by Weak Bisimilarity.
	MSCS, 29(1):67--92,
	2019.
	\item P. Dybjer and H. Moeneclaey.
	Finitary Higher Inductive Types in the Groupoid Model.
	ENTCS, 336:119--134.
	\item C. Hermida and B. Jacobs.
	Structural Induction and Coinduction in a Fibrational Setting.
	{\em Information and computation}, 145(2):107--152.
	\item K. Sojakova.
	Higher Inductive Types as Homotopy-Initial Algebras.
	In POPL 2015, pages 31--42.
	\item V. Voevodsky, B. Ahrens, D. Grayson, et~al.
	UniMath --- a computer-checked library of univalent mathematics.
	available at \url{https://github.com/UniMath/UniMath}.
\end{enumerate}
\end{frame}

\remove{
\frametitle{Extra Slide: All Endpoints}
\begin{prooftree}
	\AxiomC{$P : \poly$}
	\UnaryInfC{$\id{A} : \ep{A}{P}{P}$}
\end{prooftree}
\begin{bprooftree}
	\AxiomC{$P, Q, R : \poly$}
	\AxiomC{$e_1 : \ep{A}{P}{Q}$}
	\AxiomC{$e_2 : \ep{A}{Q}{R}$}
	\TrinaryInfC{$\comp{e_1}{e_2} : \ep{A}{P}{R}$}
\end{bprooftree}
\begin{bprooftree}
	\AxiomC{$P, Q : \poly$}
	\UnaryInfC{$\inle : \ep{A}{P}{\sumP{P}{Q}}$}
\end{bprooftree}
\begin{bprooftree}
	\AxiomC{$P, Q : \poly$}
	\UnaryInfC{$\inre : \ep{A}{Q}{\sumP{P}{Q}}$}
\end{bprooftree}
\begin{bprooftree}
	\AxiomC{$P, Q : \poly$}
	\UnaryInfC{$\prle : \ep{A}{\prodP{P}{Q}}{P}$}
\end{bprooftree}
\begin{bprooftree}
	\AxiomC{$P, Q : \poly$}
	\UnaryInfC{$\prre : \ep{A}{\prodP{P}{Q}}{Q}$}
\end{bprooftree}
\begin{bprooftree}
	\AxiomC{$\constr : \ep{A}{A}{\I}$}
\end{bprooftree}
\begin{bprooftree}
	\AxiomC{$P : \poly$}
	\AxiomC{$X : \hset$}
	\AxiomC{$x : X$}
	\TrinaryInfC{$\Ce \> x : \ep{A}{P}{\C \> X}$}
\end{bprooftree}
\begin{bprooftree}
	\AxiomC{$P, Q, R: \poly$}
	\AxiomC{$e_1 : \ep{A}{P}{Q}$}
	\AxiomC{$e_2 : \ep{A}{P}{R}$}
	\TrinaryInfC{$\pair{e_1}{e_2} : \ep{A}{P}{\prodP{Q}{R}}$}
\end{bprooftree}
}

\end{document}
